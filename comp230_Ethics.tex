% Please do not change the document class
\documentclass{scrartcl}

% Please do not change these packages
\usepackage[hidelinks]{hyperref}
\usepackage[none]{hyphenat}
\usepackage{setspace}
\doublespace

% You may add additional packages here
\usepackage{amsmath}
\usepackage{graphicx}
\usepackage{wrapfig}
\graphicspath{ {./images/} }

% Please include a clear, concise, and descriptive title
\title{Issues with crowdfunded video game projects not delivering on their promises} %  What do crowdfunded game projects actually deliver compared to what they advertise?

% Please do not change the subtitle
\subtitle{COMP230 - Ethics and Professionalism}

% Please put your student number in the author field
\author{1507516}

\begin{document}

\maketitle

\abstract{Crowdfunding has become increasingly popular as a way to fund game development in recent years, this paper reflects how games are marketed in crowdfunding and the ethics of not delivering on promises to backers.}

\section{Introduction}
Crowdfunding has become a new popular way to fund video games\cite{xu2014show}, as Kickstarter alone has provided funding for over 450 video game projects (as of 2013) \cite{Harris:2013}. %However many of those game projects are still in development and have not delivered on a lot of their promses.

This paper will compare a few recent game projects and analyze what they promised they were going to deliver and see what they actually delivered.
\begin{itemize}

\item  \textbf{Game One : Elite: Dangerous}
\item  \textbf{Game Two: Planetary Annihilation}
\item  \textbf{Game Three: Godus}
\end{itemize}

Do crowdfunded projects over scope to gain money for a project, when they don't intend to deliver?


Some crowdfunded games often increase the scope of the game after the game gets more funding, which in-turn sets back the release date of the game. A particularly egregious example of this is ``Star Citizen'' \cite{kickstarterStarCitizen}. This title asked for £2.0M in funding, but as of November 2016 it has received over £132.5M. However this means that the scope of their game has increased remarkably and pushed the release date of the game back significantly (TODO: Find source for dates).

\section{Not delivering on promises to backers?}

\section{Comparing what some games projects delivered compared to what they promised}
A study found that most kickstarted projects manage to deliver their products, around 91\% of all kickstarted projects succeed \cite{mollick2015}.

\subsection{Godus}
Godus is a game developed by 22 cans, this game has had a view controversial issues surrounding it's crowdfunding. (citation needed)


\subsection{Elite: Dangerous}


\subsection{Planetary Annihilation}


\section{Advertising too much and delivering too little? (overscoping project)}

Advertising has a large part to do with the success of a corwdfunded game \cite{qiu2013} \cite{Greenberg:2013}.  So it may be tempting when starting a crowdfunded project to overscope to try and get as much support as possible, even though the product is too ambitious. This may be partly due to novice crowdfunders thinking starting up a crowdfunding project would require less work and skill than they expect \cite{xu2014show}.

\section{Should crowdfunded games refund players for not delivering on promises?}

The backers often refer to themselves as ``investors'' in corwdfunded games, however this is not true because they do not gain a profit from the video game \cite{Harris:2013}. This means that if the game fails they do not get their money back, and in comparison with traditional games, if the game is not completed then players could ask for a refund.

 Other times backers are considered to have pre-ordered the game.

\section{Conclusion}

TODO: Conclusion.


\bibliographystyle{ieeetr}
\bibliography{comp230_Ethics}

\end{document}