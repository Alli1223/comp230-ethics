% Please do not change the document class
\documentclass{scrartcl}

% Please do not change these packages
\usepackage[hidelinks]{hyperref}
\usepackage[none]{hyphenat}
\usepackage{setspace}

\doublespace

% You may add additional packages here
\usepackage{amsmath}
\usepackage{graphicx}
\usepackage{wrapfig}
\graphicspath{ {./images/} }

% Please include a clear, concise, and descriptive title
\title{Issues with crowd-funded video game projects not delivering on their promises to backers} %  What do crowdfunded game projects actually deliver compared to what they advertise?

% Please do not change the subtitle
\subtitle{COMP230 - Ethics and Professionalism}

% Please put your student number in the author field
\author{1507516}

\begin{document}

\maketitle

\abstract{Crowd-funding has become increasingly popular as a way to fund game development in recent years, this paper reflects how games are marketed in crowd-funding and the ethics of not delivering on promises to backers.}

\section{Introduction}
Crowd-funding has become a new popular way to fund video games\cite{xu2014show}, as Kickstarter alone has provided funding for over 450 video game projects (as of 2013) \cite{Harris:2013}. %However many of those game projects are still in development and have not delivered on a lot of their promses.

This paper will compare a few recent game projects and analyze what they promised they were going to deliver and see what they actually delivered.

Do crowd-funded projects over scope to gain money for a project, when they don't intend to deliver?


Some crowd-funded games often increase the scope of the game after the game gets more funding, which in-turn sets back the release date of the game. A particularly egregious example of this is ``Star Citizen'' \cite{kickstarterStarCitizen}. This title asked for \pounds2.0M in funding, but as of November 2016 it has received over \pounds132.5M. However this means that the scope of their game has increased remarkably and pushed the release date of the game back significantly, and as of Nov-2016, they do not have an estimated release date.

\section{Are crowd-funded campaigns not delivering on promises to backers?}

A study found that most kickstarted projects manage to deliver their products, around 91\% of all kickstarted projects succeed \cite{mollick2015}. However according to Chung et al. \cite{Chung:2015} the success rate of crowd-funded campaigns is falling because of projects launching without enough preparation.

In the case of Peter Molyneux's game ``Godus'', it raised \pounds526,563 on kickstarter \cite{GodusKickstarter}, and failed to deliver on a lot of its kickstarter promises, as it was advertised as a primary PC game, with support for mobile but ended up being only available on mobile devices \cite{GodusFailure}.

Another prime example of this is a kickstarted game called ``Yogventures'' which was based around the Youtube group called the Yogscast. They raised \pounds457,482 and the game failed because they over-scoped the game, furthermore the backers would not be getting any refunds \cite{YogventuresFailure}.


\section{ Do crowd-funded games often advertise too much to try and gain popularity and end up never delivering?}

Advertising has a large part to do with the success of a crowd-funded game \cite{qiu2013, Greenberg:2013}.  It may be tempting when starting a crowd-funded project to over-scope to try and get as much support as possible, even though the product is too ambitious. This may be partly due to novice crowd-funders thinking starting up a crowd-funding project would require less work and skill than they expect \cite{xu2014show}.

\section{Should crowd-funded games be treated less like an investment?}

The backers often refer to themselves as ``investors'' in crowd-funded games, however this is not true because they do not gain a profit from the video game \cite{Harris:2013}. This means that if the game fails they do not get their money back, and in comparison with games financed by publishers, if the game is not completed then players can ask for a refund.

\section{Conclusion}

Crowd-funding is still relatively new, with little in the way of academic literature on the ethics of crowd-funding, the findings presented in this paper still requires further research to determine if video games are advertising too much in their crowd-funding campaigns.


\bibliographystyle{ieeetr}
\bibliography{comp230_Ethics}

\end{document}