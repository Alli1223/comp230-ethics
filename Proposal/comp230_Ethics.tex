% Please do not change the document class
\documentclass{scrartcl}

% Please do not change these packages
\usepackage[hidelinks]{hyperref}
\usepackage[none]{hyphenat}
\usepackage{setspace}
\doublespace

% You may add additional packages here
\usepackage{amsmath}
\usepackage{graphicx}
\usepackage{wrapfig}
\graphicspath{ {./images/} }

% Please include a clear, concise, and descriptive title
\title{Is crowdfunding having a negative effect on the games industry? If so how can this be avoided? } 

% Please do not change the subtitle
\subtitle{COMP230 - Ethics and Professionalism}

% Please put your student number in the author field
\author{1507516}

\begin{document}

\maketitle

\section{Proposal}

The topic I intend to research is on the issues of crowdfunding and how this method of funding effects the games project. For example comparing how successful they are compared to traditional publishers. 

\subsection{Paper One:}

Kickstarter As a Source of Funds for Computer Games
\cite{Harris:2013}

This paper states that while it is too early to makde definate judgemets kickstarter is not a viable alternative to traditional publishers.

Of the 468 projects analyzed, only 129(27percent) have delivered their game as of 2013.

Occasionally, a game must change features or otherwise break a promise made during funding, 
and there is not yeta clearly understood set of obligations between creators and backers for how this situation should be handled.



\subsection{Paper Two:}

Public Online Failure With Crowdfunding
\cite{Greenberg}

The majority of projects on kickstarter fail. This paper focuses on public embarrasment in the process of running a kickstarted project.

This issue is important to ethics because having a project be a public embarrasment impacts the developers of that game.




\subsection{Paper Three:}

Show me the money!: an analysis of project updates during crowdfunding campaigns
\cite{xu2014show}

This paper tries to understand the factors that affect a kickstarter campaign.

They found that specific uses of updates had stronger associations with campaign success than the project’s description.

They desgined formulas from the results to help designers better support various uses of updates in crowdfunding campaigns.

Updates are critical to the success of a campaign.

the chance of success of a project without an update was only 32.6percent.

\subsection{Paper Four:}

A Long-Term Study of a Crowdfunding Platform: Predicting Project Success and Fundraising Amount
\cite{Chung:2015}

\section{}


\section{Conclusion}


\bibliographystyle{ieeetr}
\bibliography{comp230_Ethics}

\end{document}
